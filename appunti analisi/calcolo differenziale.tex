\documentclass{article}
\usepackage{graphicx} % Required for inserting images
\usepackage{amsmath}
\usepackage{amssymb}
\usepackage{mathtools}
\usepackage{tikz,lipsum,lmodern}
\usepackage[most]{tcolorbox}
\usepackage{xcolor}
\usepackage{amsthm}
\usepackage{enumitem}


\title{Calcolo differenziale}
\author{Elisa Frigerio }
\date{11/12- 21/12 2023}

\begin{document}

\maketitle

\section{Funzioni derivabili e differenziabili}

\begin{tcolorbox}[colback = pink!10!white,   colframe=red!50!blue]
  \textcolor{violet}{DEFINIZIONE: }

  $f:=(a, b) \in \mathbb{R} \to \mathbb{R}$ è \textcolor{purple}{derivabile} in $x_0 \in (a, b)$ se

  \[ \exists \lim_{h \to 0} \frac{f(x_0 + h) -f(x_0)}{h}=\lim_{x \to x_o} \frac{f(x) -f(x_0)}{x-x_0}:=f'(x_0) \in \mathbb{R}\]

\end{tcolorbox}

\begin{tcolorbox}[colback=white, colframe=purple]
 \begin{itemize}
     \item f è \textcolor{purple}{derivabile} in (a,b) se è derivabile in \textcolor{purple}{ogni punto} di (a,b)
     \item $y=f'(x_0)(x-x_0)+f(x_0)$ è la retta \textcolor{purple}{tangente} a $f(x)$ in $x_0$
 \end{itemize}

\end{tcolorbox}


\begin{tcolorbox}[colback = pink!10!white,   colframe=red!50!blue]

  $f:=(a, b) \in \mathbb{R} \to \mathbb{R}$ è \textcolor{purple}{differenziabile} in $x_0 \in (a, b)$ se

  f si approssima con il \textcolor{purple}{polinomio} $P(x,x_0)=f(x_0) + f'(x_0)(x - x_0)$, più un \textcolor{purple}{resto} $R(x, x_0) = o(x-x_0)$ \textcolor{purple}{trascurabile} rispetto a $x - x_0$


\end{tcolorbox}

\[\lim_{x \to x_o} \frac{f(x) -f(x_0)}{x-x_0}:=f'(x_0) \in \mathbb{R}\]
\[\lim_{x \to x_o} \left[\frac{f(x) -f(x_0)}{x-x_0} - f'(x_0) \right] = 0\]
\[\frac{f(x) -f(x_0)}{x-x_0} - f'(x_0)  = o(1)\]
\[f(x) = f(x_0) + f'(x_0)(x - x_0) + o(x - x_0)\]

Per funzioni ad una variabile derivabile $\Rightarrow $ differenziabile. 

\begin{tcolorbox}[colback= red!15!yellow!5!white, colframe=red]
  \textcolor{red}{TEOREMA: }
  Una funzione \textcolor{purple}{derivabile} è \textcolor{purple}{continua}. Non vale il viceversa. 
\end{tcolorbox}

\begin{proof}

$\lim_{x \to x_0} f(x) = \lim_{x \to x_0} f(x_0) + f'(x_0)(x - x_0) + o(x - x_0) = f(x_0) \Rightarrow $ è \textcolor{purple}{continua}
    
\end{proof}

\subsection{Derivate fondamentali}
Si dimostra che: 

\begin{table}[h]
    \centering
    \begin{tabular}{|c|c|}
         f(x)& f'(x) \\
         c& 0\\
         $x^n$ & $n\cdot x^{n-1}$\\
         $x^a \; a \in \mathbb{R}, x >0 $&  $ a\cdot x^{a-1}$\\
         $\lg|x|$ &  $\frac{1}{x} \; \forall x \in \mathbb{R}\setminus \{0\}$\\
          $\sin x$ & $\cos x$\\
         $ \cos x $& $-\sin x$ \\
         $e^x$& $e^x$ \\
    \end{tabular}
\end{table}

\subsection{Formule di derivazione}

\begin{tcolorbox}[colback= red!15!yellow!5!white, colframe=red]
  \textcolor{red}{TEOREMA: }
 
Siano $f,g := (a,b) \subset \mathbb{R} \to \mathbb{R}$ derivabili in $x_0$

\[c\cdot f(x), f(x)\pm g(x), f(x)\cdot g(x), \frac{f(x)}{g(x)} \; (g(x_0)\neq 0), \]
sono \textcolor{purple}{derivabili}, e vale: 
\end{tcolorbox}

\begin{table}[h]
    \centering
    \begin{tabular}{|c|c|}
        \hline
        \\
        $c\cdot f(x_0)$ & $c \cdot f'(x_0)$\\
        \\
         \hline
         \\
         $f(x_0) \pm g(x_0)$&  $f'(x_0) \pm g'(x_0)$\\
         \\
         \hline
         \\
         $f(x_0) \cdot g(x_0)$& $ f'(x_0)g(x_0) + g'(x_0)f(x_0)$ \\
         \\
         \hline
         \\
         $\frac{f(x_0)}{g(x_0)}$&  $\frac{ f'(x_0)g(x_0) - g'(x_0)f(x_0)}{g(x_0)^2}$\\
         \\
         \hline
         
    \end{tabular}
\end{table}

      \begin{tcolorbox}[colback= red!15!yellow!5!white, colframe=red]

Siano $f:= I \to J $ derivabile in $x_0 \in I$  e $ g:= J \to \mathbb{R} $ derivabile in $f(x_0) \in J $ I, J \textcolor{purple}{intervalli }$ \Rightarrow g \circ f $ è \textcolor{purple}{derivabile} 

\end{tcolorbox}

\begin{table}[h]
    \centering
    \begin{tabular}{|c|c|}
         \hline
         \\
     $g(f(x_0)$ & $g'(f(x_0)f'(x_0)$ \\
     \\
        \hline
    \end{tabular}
    
\end{table}

\subsection{Derivata dell'inversa}

      \begin{tcolorbox}[colback= red!15!yellow!5!white, colframe=red]
  \textcolor{red}{TEOREMA: }
 
Sia $f:= I \to J $ \textcolor{purple}{invertibile} e sia $f^{-1}:= J \to I$ la sua \textcolor{purple}{inversa}

Se f è \textcolor{purple}{derivabile} in $x_0 \Rightarrow f^{-1}$ è \textcolor{purple}{derivabile } in $y_0=f(x_0)$ e vale 
\[[f^{-1}(y_0)]' = \frac{1}{f'(x_0)}\]


\end{tcolorbox}

\subsection{Punti di non derivabilità}

Si dice che $f$, continua in $x_0$ presenta in $x_0$: 

\begin{itemize}
    \item \begin{tcolorbox}[colback = pink!10!white,   colframe=red!50!blue]
    Un \textcolor{purple}{punto angoloso } se: 
    \[f'_+(x_0) \in \mathbb{R} \neq f'_-(x_0) \in \mathbb{R} \]
    \end{tcolorbox}
    \item \begin{tcolorbox}[colback = pink!10!white,   colframe=red!50!blue]
    Una \textcolor{purple}{cuspide} se: 
    \[\lim_{h \to 0}\frac{f(x_0 + h) -f(x_0)}{h} = l \in \{- \infty, + \infty\} \neq \lim_{h \to 0}\frac{f(x_0 + h) -f(x_0)}{h} = L \in \{- \infty, + \infty\} \]
    \end{tcolorbox}
     \item \begin{tcolorbox}[colback = pink!10!white,   colframe=red!50!blue]
    Un \textcolor{purple}{punto a tangente verticale} se: 
    \[\lim_{h \to 0}\frac{f(x_0 + h) -f(x_0)}{h} = l \in \{- \infty, + \infty\} = \lim_{h \to 0}\frac{f(x_0 + h) -f(x_0)}{h} = l \in \{- \infty, + \infty\} \]
    \end{tcolorbox}
\end{itemize}

\section{Applicazioni delle derivate}

\subsection{Estremi}

\begin{tcolorbox}[colback = pink!10!white,   colframe=red!50!blue]
  \textcolor{violet}{DEFINIZIONE: }
  Sia $f:= I \to \mathbb{R} $, I intervallo $x_0 \in I$

  $x_0 $ è un punto di
  \begin{itemize}
      \item \textcolor{purple}{massimo locale} (o relativo)  se $\exists \delta >0 \mid f(x_0) \ge f(x) \forall x \in I \cap (x_0 - \delta, x_0 + \delta)$
      \item \textcolor{purple}{minimo locale} (o relativo) se $\exists \delta >0 \mid f(x_0) \le f(x) \forall x \in I \cap (x_0 - \delta, x_0 + \delta)$
  \end{itemize}
\end{tcolorbox}
\begin{itemize}
    \item $f(x_0)$ è detta rispettivamente \textcolor{purple}{massimo o minimo} locale.
    \item $x_0$ si dice \textcolor{purple}{estremante relativo}, $f(x_0)$ si dice \textcolor{purple}{estremo relativo} 
    \item se la proprietà vale su \textcolor{purple}{tutto }I, si dice che l'\textcolor{purple}{estremante} è \textcolor{purple}{globale}(o assoluto)
    \item Se la disuguaglianza è \textcolor{purple}{stretta}, si dice massimo o minimo \textcolor{purple}{forte}
\end{itemize}

\subsection{Teorema di Fermat}

      \begin{tcolorbox}[colback= red!15!yellow!5!white, colframe=red]

Sia $f:= (a,b) \to \mathbb{R} $ e $x_0 \in (a,b)$ sia un \textcolor{purple}{estremo relativo}.

Se f è \textcolor{purple}{derivabile} in $x_0 \Rightarrow f'(x_0) = 0$

\end{tcolorbox}

\begin{proof} 
($x_0$ punto di \textcolor{purple}{massimo})
\[f(x_0) \ge f(x) \forall x \in I \cap (x_0 - \delta, x_0 + \delta) \Rightarrow f(x)-f(x_0) \le 0\]
    \[lim_{x \to x_0+} \frac{f(x) - f(x_0)}{x - x_0}= f'(x_0)^+ \le 0 \quad (x -x_0 > 0) \]
    \[lim_{x \to x_0-} \frac{f(x) - f(x_0)}{x - x_0}= f'(x_0)^- \ge 0 \quad (x -x_0 < 0)  \Rightarrow\]
    \[lim_{x \to x_0} \frac{f(x) - f(x_0)}{x - x_0}= f'(x_0)=0\]
    
\end{proof}

\subsection{Teorema di Rolle}

      \begin{tcolorbox}[colback= red!15!yellow!5!white, colframe=red]

Sia $f:= [a,b] \to \mathbb{R} $, 
\begin{itemize}
    \item \textcolor{purple}{continua} su $[a,b]$
    \item \textcolor{purple}{derivabile} su $(a,b)$
    \item f(a) =f(b) 
\end{itemize}

    $\Rightarrow \exists x_0 \in (a, b) \mid f'(x_0) = 0 $ 
\end{tcolorbox}

\begin{proof}
    Ho  due casi: 
    
    \begin{itemize}
    \item $f(x) = f(a) = f(b) \forall x \Rightarrow f'(x) = 0 \; \forall x$
    \item $f(x)$ non è costante:
    \end{itemize}
   Per il \textcolor{purple}{teorema di Weierstrass}, $\exists x_m \in [a,b] \mid x_m$ è un \textcolor{purple}{punto di minimo assoluto}, $\exists x_M \in [a,b] \mid x_M$ è un \textcolor{purple}{punto di massimo assoluto}. 

   Siccome $f(a) = f(b) $ e $f(x_m )\neq f(x_M)$, non possono cadere entrambi sugli estremi. Sia $x_0 $ il punto che cade in $(a,b)$, per il \textcolor{purple}{teorema di Fermat}, $f'(x_0) = 0$
\end{proof}

\subsection{Teorema di Lagrange}

      \begin{tcolorbox}[colback= red!15!yellow!5!white, colframe=red]

Sia $f:= [a,b] \to \mathbb{R} $, 
\begin{itemize}
    \item \textcolor{purple}{continua} su $[a,b]$
    \item \textcolor{purple}{derivabile} su $(a,b)$
\end{itemize}

    $\Rightarrow \exists x_0 \in (a, b) \mid f'(x_0) = \frac{f(b) - f(a)}{b -a}$ 
\end{tcolorbox}

\begin{proof}
    Si considera $F(x) = f(x) - \left[ f(a) + \frac{f(b) - f(a)}{b -a}(x-a)\right] $ continua e derivabile come f. 

    Sostituendo, si ottiene: $F(a) =F(b)=0$, che soddisfa le ipotesi per il \textcolor{purple}{teorema di Rolle}

$\Rightarrow \exists x_0 \mid F'(x_0) =0$
\[F'(x_0)= f'(x_0) - \frac{f(b) - f(a)}{b -a} = 0 \]
\[f'(x_0) = \frac{f(b) - f(a)}{b -a} \]

\end{proof}

\subsection{Teorema di Cauchy }


 \begin{tcolorbox}[colback= red!15!yellow!5!white, colframe=red]

Sia $F,G:= [a,b] \to \mathbb{R} $, 
\begin{itemize}
    \item \textcolor{purple}{continue} su $[a,b]$
    \item \textcolor{purple}{derivabili} su $(a,b)$
\end{itemize}
$\exists x_0 \in (a, b) \mid $

\[ [G(b)-G(a)] F'(x_0) = [F(b)-F(a)]G'(x_0)\]
\end{tcolorbox}

\begin{proof}
    Sia $f(x) = [G(b)-G(a)] F(x) - [F(b)-F(a)]G(x)$

    E' evidente che $f(a) = f(b) = 0$ : f, (che eredita le proprietà di \textcolor{purple}{continuità  e derivabilità } di F e G) soddisfa le ipotesi del \textcolor{purple}{teorema di Rolle}

    $\exists x_0 \mid f'(x_0) = 0$, cioè $[G(b)-G(a)] F'(x_0) - [F(b)-F(a)]G'(x_0) = 0$   


    
    
\end{proof}

\subsection{Teorema di De l'Hopital}

\begin{tcolorbox}[colback= red!15!yellow!5!white, colframe=red]

Sia $f,g:= [a,b] \to \mathbb{R} $, 
\begin{itemize}
\item \[ \lim_{x \to a^+} f(x)= \lim_{x \to a^+} g(x)= l \in \{- \infty, 0, + \infty\}\]
    \item \textcolor{purple}{derivabili} in $(a,b)$
    \item $g(x), g'(x) \neq 0 \forall x \in (a, b) $
    \item $\exists \lim_{x \to a^+}\frac{f'(x)}{g'(x)}= L \in \overline{\mathbb{R}}$
\end{itemize}
$\Rightarrow  \lim_{x \to a^+}\frac{f(x)}{g(x)}= L$




\end{tcolorbox}

Il teorema vale anche per $x \to b^-$

\begin{proof}

Caso $l=0, L\in \mathbb{R}$ 

Per ipotesi: $\lim_{x \to a^+}\frac{f'(x)}{g'(x)}= L \in \overline{\mathbb{R}} \Rightarrow$

\[\forall \epsilon>0 \exists \delta=\delta(\epsilon) \mid L- \epsilon < \frac{f'(x)}{g'(x)} < L + \epsilon \]

Per ogni coppia $x, \mu \in (a, a + \delta) \exists x<z<\mu \mid $

\[ L- \epsilon < \frac{f(x)-f(\mu)}{g(x) - g(\mu)}=\frac{f'(z)}{g'(z)} < L + \epsilon \]

Per $\mu \to a^+, f(\mu), g(\mu) \to 0$ (l=0) per ipotesi

\[ L- \epsilon < \frac{f(x)}{g(x)}=\frac{f'(z)}{g'(z)} < L + \epsilon \]

$\Rightarrow  \lim_{x \to a^+}\frac{f(x)}{g(x)}= L$


\end{proof}



\subsection{Monotonia}

Sia $f:= I \to \mathbb{R} $, I intervallo, f \textcolor{purple}{derivabile} in I: 

  \begin{tcolorbox}[colback= red!15!yellow!5!white, colframe=red]
  \textcolor{red}{TEOREMA: }


\begin{itemize}
    \item f \textcolor{purple}{crescente} in I $\Leftrightarrow f'(x) \ge 0 \forall x \in I$
    \item f \textcolor{purple}{decrescente} in I $\Leftrightarrow f'(x) \le 0 \forall x \in I$
    
\end{itemize}

\end{tcolorbox}

\begin{proof}
(f crescente)
    ($\Leftarrow$) 
    Siano $x_1, x_2 \in I \mid x_1 < x_2$
    Considero $f:=[x_1,x_2] \to \mathbb{R}$ e applico Lagrange: 
    \[\exists x_0 \mid f'(x_0)= \frac{f(x_2)-f(x_1)}{x_2 -x_1} \\ f(x_2) -f(x_1) = f'(x_0) (x_2-x_1) \ge 0 \Rightarrow f(x_2) \ge f(x_1) \]

    
    ($\Rightarrow$) 
    \textcolor{red}{Per assurdo}, se $ \exists x_0 \in I f'(x_0) < 0$, 
    poichè f è \textcolor{purple}{derivabile} in I si ha: 
    

    \[f(x)- f(x_0) =   (x - x_0)[f'(x_0) + o(1)] \]

    Se $x < x_0 \Rightarrow f(x) > f(x_0) $, ma $f(x) \le f(x_0) \Rightarrow$ \textcolor{red}{assurdo}

    Analogamente per f decrescente
    
\end{proof}
    Corollari: 
\begin{tcolorbox}[colback=white, colframe=purple]
\begin{itemize}
    \item se $f'(x_0)>0 \Rightarrow $ f è \textcolor{purple}{strettamente crescente}
    \item se $f'(x_0)<0 \Rightarrow $ f è \textcolor{purple}{strettamente decrescente}
\end{itemize}

\end{tcolorbox}
\textcolor{purple}{NON} vale $\Leftarrow$


\begin{tcolorbox}[colback=white, colframe=purple]
f è \textcolor{purple}{costante} $ \Leftrightarrow f'(x)=0 \quad \forall x \in I $
\end{tcolorbox}
\begin{proof}
    \begin{math} f'(x)=0 \Rightarrow
        \begin{cases}
            f'(x) \ge 0 \; \Leftrightarrow f(x9 \text{ è crescente} \\
            f'(x) \le 0 \; \Leftrightarrow f(x9 \text{ è decrescente}
        \end{cases} \Rightarrow
    \end{math}
    f(x) è \textcolor{purple}{costante}
\end{proof}

\begin{tcolorbox}[colback=white, colframe=purple]
    Se $f'(x_0 \ge 0 \forall x \in I$ e $E:=\{x \in I\mid f'(x)=0 \} = \emptyset $ o ha \textcolor{purple}{solo punti isolati}, allora f è \textcolor{purple}{strettamente crescente}
Analogamente, se $f'(x_0)<0 \Rightarrow $ $I:=\{x \mid f'(x)=0 \} = \emptyset $ o ha \textcolor{purple}{solo punti isolati}, allora f è \textcolor{purple}{strettamente decrescente}  
\end{tcolorbox}


\subsubsection{Massimi e minimi}

Sia $f:=(x_0 - \delta, x_o + \delta) \in \mathbb{R} quad \delta >0$, derivabile in $(x_0 - \delta, x_o + \delta) $

$x_0$ è 
\begin{itemize}
    \item \textcolor{purple}{massimo stretto relativo} se
    \[f'(x_0) < 0 \; \forall x \in (x_0 - \delta, x_0) \quad f'(x_0) > 0 \; \forall x \in (x_0, x_0 + \delta) \]
     \item \textcolor{purple}{minimo stretto relativo} se
    \[f'(x_0) > 0 \; \forall x \in (x_0 - \delta, x_0) \quad f'(x_0) < 0 \; \forall x \in (x_0, x_0 + \delta) \]
\end{itemize}


\section{Applicazioni della derivata seconda}

Sia $f:= (a,b)\to \mathbb{R}$, f derivabile in \textcolor{purple}{intorno} di $x_0$
\begin{tcolorbox}[colback = pink!10!white,   colframe=red!50!blue]
    si definisce \textcolor{purple}{derivata seconda}
    \[f''(x_0):= \lim_{x \to x_0} \frac{f'(x) -f'(x_0)}{x - x_0}\]
\end{tcolorbox}

\subsection{Ricerca estremi relativi}

Sia $f:=(a,b9 \to \mathbb{R} \quad x_0 \in (a,b)$ \textcolor{purple}{due} volte derivabile in $x_0$

Se $f'(x_0) = 0$ e

\begin{itemize}
    \item $f''(x_0)>0 \Rightarrow x_0$ è \textcolor{purple}{minimo} relativo forte
    \item $f''(x_0)<0 \Rightarrow x_0$ è \textcolor{purple}{massimo} relativo forte 
    
\end{itemize}

Sono condizioni \textcolor{purple}{solo sufficienti}

Per \textcolor{purple}{Taylor}, per $x \to x_0$ ($f'(x_0)=0)$

\[f(x)= f(x_0) + \frac{f''(x_0)}{2!}(x-x_0)^2 + o[x-x_0)^2]\]

La funzione è approssimabile ad una \textcolor{purple}{parabola} con vertice in $(x_0, f(x_0)$ e \textcolor{purple}{concavità} data dal segno di $f''(x_0)$

\subsection{Concavità e convessità}

\subsubsection{Definizione di convessità}

Sia $f:= I \to \mathbb{R}$, I un generico intervallo
\begin{tcolorbox}[colback = pink!10!white,   colframe=red!50!blue]

    f è detta: 

    \begin{itemize}
        \item \textcolor{purple}{convessa} in I se, per ogni \textcolor{purple}{terna} $x_1 < x< x_2$
        \[f(x) \le f(x_1) + \frac{f(x_2) - f(x_1)}{x_2 - x_1}(x -x_1)\]
        ossia la funzione si trova \textcolor{purple}{sotto} la retta \textcolor{purple}{secante}
        \item \textcolor{purple}{concava} in I se, per ogni \textcolor{purple}{terna} $x_1 < x< x_2$
        \[f(x) \ge f(x_1) + \frac{f(x_2) - f(x_1)}{x_2 - x_1}(x -x_1)\]
        ossia la funzione si trova \textcolor{purple}{sopra} la retta \textcolor{purple}{secante}
    \end{itemize}
  
\end{tcolorbox}

\subsubsection{Convessità e derivata seconda}

Sia $f:= I \to \mathbb{R}$, I generico intervallo

Sono \textcolor{purple}{equivalenti} le seguenti affermazioni: 

\begin{enumerate}[label=(\alph*)]
    \item f è \textcolor{purple}{convessa} secondo la definizione data sopra 
    \item Per ogni coppia di punti $x_0, x \in I$
    \[f(x) \ge  f(x_0) + f'(x_0)(x - x_0)\]
    (Ossia, la funzione si trova \textcolor{purple}{sopra} la retta \textcolor{purple}{tangente}
    \item $f''(x) \ge 0 \quad \forall x \in I$
    \item  $f'(x)$ è \textcolor{purple}{crescente} in I  
\end{enumerate}

Allo stesso modo, per f \textcolor{purple}{concava}
\section{Formule di Taylor}

 \begin{tcolorbox}[colback= red!15!yellow!5!white, colframe=red]
  \textcolor{red}{TEOREMA: }
  
Sia $f:(a,b) \to \mathbb{R}$ derivabile \textcolor{purple}{n volte} in $x_0$ (derivabile n-1 volte in $B_{\delta}(x_0)$ e derivabile in $x_0$, 

per $x \to x_0$ 

\[f(x)= P_{n}^f(x, x_0) + R_{n}^f(x, x_0) \]

\end{tcolorbox}


con P, \textcolor{purple}{polinomio di Taylor}, tale che
\[P_{n}^f(x,x_0) = f(x_0) +f'(x_0)(x - x_0) + \frac{f''(x_0)}{2!}(x-x_0)^2 + ...  \frac{f^{[n]}(x_0)}{n!}(x-x_0)^n \]

e con R \textcolor{purple}{Resto di Peano}, tale che 

\[ R_{n}^f(x,x_0)= o(x-x_0)^n \]


Se $x \to 0$, le precedenti formule si dicono \textcolor{purple}{formule di Mc. Laurin}

\begin{proof}
    Il resto di Peano al II ordine è $o(x-x_0)$

Dobbiamo dimostrare che: 
    \[\lim_{x\to x_o}\frac{R_{2,f}(x,x_0)}{(x-x_0)^2}=0 \]
    
    \[\Rightarrow \lim_{x\to x_o}\frac{f(x)-P_{2,f}(x,x_0)}{(x-x_0)^2}=0\]

    Presenta la forma di indecisione $\frac{0}{0}$, applichiamo quindi il \textcolor{purple}{teorema di De l'Hopital}

    \[\lim_{x\to x_o}\frac{f'(x)-[f'(x_0)+2f''(x_0)(x-x_0)]}{2(x-x_0)}=\lim_{x\to x_o}\frac{1}{2} \left[ \frac{f'(x)-f'(x_0)]}{(x-x_0)} - f''(x_0) \right]= \] 
    \[\lim_{x\to x_o}\frac{1}{2} [f''(x) -f''(x_0) =0\]
\end{proof}

\subsubsection{Unicità dello sviluppo}

\begin{tcolorbox}[colback= red!15!yellow!5!white, colframe=red]
  \textcolor{red}{TEOREMA: }

Sia $f:=(a,b) \to \mathbb{R}$ e $x_0 \in (a, b)$

Supponiamo che valgano per $x \to x_0$: 
\[f(x)=a_0 + a_1(x-x_0)+...+a_n(x-x_0)^n + o[(x-x_0)^n]\]
\[f(x)=b_0 + b_1(x-x_0)+...+b_n(x-x_0)^n + o[(x-x_0)^n]\]

$\Rightarrow a_k = b_k \forall k \in 0,..,n$
\end{tcolorbox}

\begin{proof}
    \textcolor{red}{Per assurdo: } $\exists k \le n \mid a_k \neq b_k $

    \begin{itemize}
        \item Sottraggo il II sviluppo al primo e ottengo: 
        \[0=(a_k-b_)(x -x_0)^k + ... + (a_n - b_n)(x - x_0)^n + o[(x-x_0)^n]\]
        \item per $x \neq x_0$ divido per $(x - x_0)^k$
        \[-a_k + b_k = ... + (a_n - b_n) (x - x_0)^{n-k} + o[(x-x_0)^{n-k}]\]
        \item Passando al limite per $x \to x_0$
        \[\lim_{x \to x_0} -a_k + b_k =-a_k + b_k = 0  \]
    \end{itemize}

    $\Rightarrow a_k = b_k$
    
\end{proof}

\begin{tcolorbox}[colback=white, colframe=purple]
    Se $f:(a,b) \to \mathbb{R} \quad x_0 \in (a,b)$ è \textcolor{purple}{derivabile} n-volte in $x_0$ e $f(x)=a_0 + a_1(x-x_0)+...+a_n(x-x_0)^n + o[(x-x_0)^n]$, 

    \begin{itemize}
        \item $a_0 = f(x_0)$
        \item $a_1= f'(x_0)$
        \item ...
        \item $a_n = \frac{f^{[n]}(x_0)}{n!}$
    \end{itemize}

    \[f^{[n]}(x_0) = a_n*n!\]
\end{tcolorbox}



\end{document}
